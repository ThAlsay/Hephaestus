\chapter{Análisis}\label{Cap4}

El proyecto cuenta con tres grandes bloques de contenido con código: el motor Hephaestus, el juego didáctico y las prácticas para los alumnos. El código de las prácticas para los alumnos, al tratarse de las soluciones de estas, es sencillo y orientativo, por lo tanto queda excluido del análisis formal del proyecto.

\section{Motor Hephaestus}

Se trata del foco principal del proyecto, por lo tanto el grueso del análisis se concentra en el motor de videojuegos distribuidos Hephaestus.

\subsection{Requisitos}

Hephaestus se describe como un motor para el desarrollo de videojuegos distribuidos escrito en Elixir. Por lo tanto dicho motor deberá contar, fundamentalmente, con: la capacidad de ser distribuido; la posibilidad de emplearlo para manejar el almacenamiento del estado del juego, ya sea de manera permanente o efímera; la capacidad de simplificar el desarrollo de videojuegos y la modularidad suficiente para admitir futuras extensiones de capacidades.

Con esta descripción como base, a continuación se muestran las tablas de requisitos funcionales y no funcionales del motor Hephaestus.


\begin{tabularx}{\textwidth}{|X|}	
	\hline
	\rowcolor[HTML]{C0C0C0} 
	\textbf{Requisito} \\ \hline
	Tiene que tener la capacidad de autenticar y autorizar a los usuarios. \\ \hline
	Cada pieza que actúe como actor debe tener un identificador único que sirva para su identificación en todo el sistema. \\ \hline
	Tiene que permitir la creación de los siguientes actores: Personaje, conexión, enemigo, ubicación, NPC y objeto. \\ \hline
	Los actores personaje, NPC y enemigo tienen que tener las siguientes estadísticas: Carisma, sabiduría, inteligencia, constitución, destreza y fuerza. \\ \hline
	Todos los actores tienen que poder devolver el estado actual en el que se encuentran. \\ \hline
	El actor personaje tiene que poder almacenar objetos en un inventario. \\ \hline
	El actor personaje tiene que poder equiparse objetos de su inventario para aumentar o disminuir sus estadísticas. \\ \hline
	El actor personaje tiene que poder subir de nivel tras llegar a un umbral de experiencia conseguida. \\ \hline
	El actor personaje, el actor NPC y el actor enemigo deben contar con un nivel con el fin de realizar resoluciones de conflictos. \\ \hline 
	El actor personaje y el actor enemigo deben contar con un contador de vida que al llegar a cero cambie su estado a no vivo pero no destruya el actor. \\ \hline
	El actor personaje tiene que indicar si se encuentra en combate o no. \\ \hline
	El actor personaje tiene que tener al capacidad de recordar la ubicación en la que se encuentra y cambiarla si se requiere. \\ \hline
	Tiene que tener los mecanismos necesarios para asociar un personaje a un usuario. \\ \hline
	El actor enemigo tiene que proporcionar una cantidad de experiencia arbitraria si su vida a llegado a cero. \\ \hline
	El actor enemigo tiene que contar con una estadística a partir de la cual se derive el daño de ataque que tiene. \\ \hline
	El actor NPC tiene que contar con un resultado para una interacción exitosa. \\ \hline
	El actor NPC tiene que contar con un límite de interacciones que pueda ser opcional. En caso de no ser opcional, el resultado de obtener la interacción debe ser exitoso hasta que se consuman todos los intentos. \\ \hline
	El actor conexión debe almacenar exactamente dos ubicaciones. \\ \hline
	El actor conexión debe mostrar la siguiente ubicación en función de la ubicación actual. \\ \hline
	El actor conexión debe tener un nivel fijo. \\ \hline
	El actor conexión debe tener la capacidad de tener un objeto que se pueda emplear desde el juego para no permitir el cruce hasta la obtención del objeto. \\ \hline
	El actor objeto tiene que tener un nivel fijo. \\ \hline
	El actor objeto tiene que tener el tipo de objeto del que se trata. \\ \hline
	El actor objeto tiene que tener un valor numérico que represente diferentes cosas dependiendo del tipo de objeto. \\ \hline
	El actor objeto tiene que tener al opción de añadir una estadística sobre la que actúe el objeto. \\ \hline
	El estado del juego se tiene que poder almacenar de forma permanente en una base de datos. \\ \hline
	Tiene que proporcionar utilidades para simplificar el enrutamiento de actores en arquitecturas distribuidas. \\ \hline
	Tiene que proporcionar utilidades para la supervisión de procesos. \\ \hline
	
	\caption{Hephaestus. Requisitos funcionales}
	\label{tab:hephaestus-req-func} \\
\end{tabularx}

\begin{tabularx}{\textwidth}{|X|}
	\hline
	\rowcolor[HTML]{C0C0C0}
	\textbf{Requisito} \\ \hline
	Tiene que estar escrito en el lenguaje de programación Elixir. \\ \hline
	Tiene que estar especializado en juegos de rol en mundos de fantasía. \\ \hline
	Las contraseñas deben ser almacenadas empleando una función hash. \\ \hline
	Tiene que estar pensado para su futura modificación haciendo uso de una arquitectura de módulos. \\ \hline
	La definición de tipos de objetos tiene que ser dependiente del juego que emplea el motor. \\ \hline
	La base de datos empleada debe ser Postgresql. \\ \hline
	El control de los actores debe residir en el motor. \\ \hline
	
	\caption{Hephaestus. Requisitos no funcionales}
	\label{tab:hephaestus-req-nfunc} \\
\end{tabularx}

\subsection{Modelos}

En esta sección se muestran los diagramas del análisis llevado a cabo sobre el motor. En primer lugar el análisis inicial muestra el análisis completo del motor.

\begin{figure}[h]
	\centering
	\includegraphics[width=0.9\textwidth]{tex_files/mainmatter/chapter_4/images/diagrama_analisis_motor/analisis_inicial_motor.png}
	\caption{Diagrama de análisis Hephaestus completo}
	\label{fig:chap4_motor_completo}
\end{figure}

De este análisis inicial del motor completo, se tienen que eliminar algunos elementos, que no suponen la imposibilidad de crear juegos sin ellos, debido a la falta de tiempo que supone la implementación completa del motor, además del resto del proyecto. Los elementos eliminados son los siguientes:

\begin{itemize}
	\item Acción.
	\item Coordinador.
	\item Acción conjunta.
	\item Combate.
	\item Interacción.
	\item Conflicto.
\end{itemize}

Una vez eliminados estos componentes el diagrama final de análisis del motor Hephaestus queda de la siguiente forma.

\begin{figure}[h]
	\centering
	\includegraphics[width=0.9\textwidth]{tex_files/mainmatter/chapter_4/images/diagrama_analisis_motor/analisis_final_motor.png}
	\caption{Diagrama de análisis Hephaestus}
	\label{fig:chap4_motor_final}
\end{figure}

\section{Juego didáctico}

Al tratarse de un foco menor respecto a Hephaestus, el análisis del juego didáctico no es tan extenso y profundo como lo es el del motor. Pese a ello, el análisis es suficiente para que el diseño posterior cumpla con las condiciones suficientes para el correcto funcionamiento de las prácticas.

\subsection{Requisitos}

El juego didáctico se caracteriza por actuar como servidor para clientes creados por los alumnos, mediante los cuales se puede resolver el juego. Es fundamental que se adapte a las necesidades del curso en el que se emplea.

Por lo tanto las siguientes listas muestran los requisitos funcionales y no funcionales del juego didáctico.

\begin{tabularx}{\textwidth}{|X|}
	\hline
	\rowcolor[HTML]{C0C0C0}
	\textbf{Requisito} \\ \hline
	Tiene que implementar un servidor de tipo TCP que admita comunicaciones concurrentes. \\ \hline
	El cliente tiene que poder conectarse empleando el puerto 3000 TCP. \\ \hline
	Tiene que admitir, únicamente, comunicaciones que empleen el protocolo JSON-RPC. \\ \hline
	Tiene que controlar todos los actores de tipo personaje y enemigo de forma remota. \\ \hline
	Tiene que definir las rutas de las máquinas en la que los actores de tipo personaje y enemigo se van a ejecutar. \\ \hline
	Tiene que contar con los suficientes procedimientos remotos para ser completado. \\ \hline
	Tiene que contar con un procedimiento remoto que devuelva la información correcta para la ejecución del resto de procedimientos. \\ \hline
	Toda gestión de la base de datos se tiene que delegar al motor Hephaestus. \\ \hline
	Los procedimientos tienen que delegar la gestión de estados de los actores al motor Hephaestus. \\ \hline
	
	\caption{Juego didáctico. Requisitos funcionales}
	\label{tab:juego-req-func}
\end{tabularx}

\begin{tabularx}{\textwidth}{|X|}
	\hline
	\rowcolor[HTML]{C0C0C0}
	\textbf{Requisito} \\ \hline
	Tiene que estar escrito en el lenguaje de programación Elixir. \\ \hline
	Tiene que hacer uso del motor Hephaestus. \\ \hline
	La definición del juego tiene que estar escrita en JSON. \\ \hline
	Su longitud tiene que estar adaptada para el contexto de la asignatura en el que es utilizado. \\ \hline
	Tiene que desarrollarse sobre un mundo compartido. Esto es, en caso de que se conecten dos personajes, las acciones de uno de los personajes sobre el mundo afecta al mundo que ve el otro personaje, es decir, es el mismo mundo. \\ \hline
	Tiene que implementar un procedimiento que ejecute un guardado del estado del juego de manera permanente. \\ \hline
	Tiene que implementar un procedimiento para la autenticación de usuarios. \\ \hline
	
	\caption{Juego didáctico. Requisitos no funcionales}
	\label{tab:juego-req-nfunc}
\end{tabularx}

\subsection{Modelos}

Teniendo en cuenta las características y los requisitos anteriores. El análisis completo del juego didáctico se muestra en el siguiente diagrama.

\begin{figure}[h]
	\centering
	\includegraphics[width=0.9\textwidth]{tex_files/mainmatter/chapter_4/images/diagrama_analisis_juego/analisis_final_juego.png}
	\caption{Diagrama de análisis juego didáctico}
	\label{fig:chap4_juego_final}
\end{figure}
